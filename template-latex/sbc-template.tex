\documentclass[12pt]{article}

\usepackage{sbc-template}

\usepackage{graphicx,url}

%\usepackage[brazil]{babel}   
\usepackage[latin1]{inputenc}  

     
\sloppy

\title{\\ Natural Language Processing\\ A study of Language Models using word2vec}

\author{Isabel B. Amaro\inst{1}, Adriano Veloso\inst{2}}


\address{Department of Computer Science -- Universidade Federal de Minas Gerais
  (UFMG)\\
  Belo Horizonte, Brazil
  \email{\{isabel.amaro,adrianov\}@dcc.ufmg.br}
}

\begin{document} 

\maketitle

\begin{abstract}
\end{abstract}

\section{Introduction}

\hspace{6ex} With the appearing of social networks, a massive amount of data is constantly being generated weekly in the last few years (CAMBRIA, 2014). Natural Language Processing is a Computer Science field that studies the best ways to retrieve, process and generate from representations of human language (CAMBRIA, 2014). To

This work analyses 

\section{Language Models}

\section{word2vec}

Word2vec can utilize either of two model architectures to produce a distributed representation of words: continuous bag-of-words (CBOW) or continuous skip-gram. In the continuous bag-of-words architecture, the model predicts the current word from a window of surrounding context words. The order of context words does not influence prediction (bag-of-words assumption). In the continuous skip-gram architecture, the model uses the current word to predict the surrounding window of context words. The skip-gram architecture weighs nearby context words more heavily than more distant context words.[1][4] According to the authors' note,[5] CBOW is faster while skip-gram is slower but does a better job for infrequent words.

\section{Continuous Bag-Of-Words (CBOW)}

\section{Skip-Gram}

\section{Result and analysis}

\section{Conclusion}

%\section{References}

%Bibliographic references must be unambiguous and uniform.  We recommend giving
%the author names references in brackets, e.g. \cite{knuth:84},
%\cite{boulic:91}, and \cite{smith:99}.

%The references must be listed using 12 point font size, with 6 points of space
%before each reference. The first line of each reference should not be
%indented, while the subsequent should be indented by 0.5 cm.

\bibliographystyle{sbc}
\bibliographystyle{unsrt}
\bibliography{sbc-template}

\bibliographystyle{unsrt}%Used BibTeX style is unsrt
\bibliography{sample}

\end{document}
